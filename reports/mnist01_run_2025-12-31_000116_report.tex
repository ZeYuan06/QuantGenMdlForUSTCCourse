% !TEX program = xelatex
\documentclass[11pt,a4paper]{article}

\usepackage{geometry}
\geometry{margin=1in}

% Robust Chinese support (uses TeX Live fonts by default if system fonts missing).
\usepackage[UTF8]{ctex}

\usepackage{amsmath,amssymb}
\usepackage{booktabs}
\usepackage{graphicx}
\usepackage{hyperref}
\hypersetup{colorlinks=true,linkcolor=blue,urlcolor=blue,citecolor=blue}

\title{MNIST01 量子生成模型对比实验报告(QDDPM / QDT / QGAN)}
\author{}
\date{}

\begin{document}
\maketitle

\noindent
实验目录:\texttt{data/mnist01\_run\_2025-12-31\_000116}\\
图表输出:\texttt{data/mnist01\_run\_2025-12-31\_000116/plots/}

\section{实验目的}
在 MNIST01(数字 0/1)任务上,将图像编码为 8-qubit 的 product Ry 量子态分布,比较三种量子生成模型 \textbf{QDDPM / QDT / QGAN} 的生成质量。

本实验强调以\textbf{量子态分布指标}为主进行评估,辅助使用分类器/图像侧指标作为补充解释。

\section{数据与实验设置}
\begin{itemize}
  \item 数据集:MNIST 0/1。
  \item 编码:product Ry 编码(目标真实态分布接近 product 态流形)。
  \item 量子比特数:$n=8$。
  \item 评估样本数:$n_{real}=2000$,$n_{gen}=2000$。
  \item 随机种子:seed=42(见各模型 \texttt{eval\_report.json})。
\end{itemize}

为保证可比性:三种模型在\textbf{同一次 run}下训练/生成并用相同脚本评估。

\section{方法简介}
\begin{itemize}
  \item \textbf{QDDPM}:量子扩散模型,学习从噪声逐步反演到数据分布。
  \item \textbf{QDT}:量子去噪/反演式生成模型(以分布距离为训练目标)。
  \item \textbf{QGAN}:生成器与判别器对抗训练的量子生成模型。
\end{itemize}

\subsection{Product Ry 投影(关键工程约束)}
MNIST01 的 qstates 是 product Ry 编码态,因此对训练与生成启用一致的结构约束:

\begin{itemize}
  \item 训练损失内投影:用测得的 $\langle Z_i\rangle$ 计算 $\theta_i=\arccos(\langle Z_i\rangle)$,并重建 product 态
  \[
    |\psi\rangle=\bigotimes_i\left(\cos(\theta_i/2)|0\rangle+\sin(\theta_i/2)|1\rangle\right).
  \]
  \item 生成输出投影:保存前同样投影到 product Ry 流形。
\end{itemize}

此举的目的:避免模型跑到强纠缠流形(这会导致单比特 purity 逼近 0.5,并使 qstates$\to$image 解码假设失效)。

\section{指标定义与含义}
\subsection{量子态主指标(主结论依据)}
\begin{itemize}
  \item \textbf{natural distance}:\texttt{qstate\_metrics.natural\_distance}(越小越好)
  \\含义:在量子态空间直接比较生成分布与真实分布的距离。
  \item \textbf{Z/ZZ 特征 MMD}:\texttt{qstate\_metrics.feature\_mmd\_rbf\_z\_zz}(越小越好)
  \\含义:以 $\langle Z_i\rangle$ 与 $\langle Z_iZ_j\rangle$ 作为特征,RBF-MMD 衡量两分布差异。
  \item \textbf{单比特 purity 均值}:\texttt{qstate\_metrics.single\_qubit\_purity\_mean}(越接近 1 越好)
  \\含义:单比特约化态纯度均值 $\mathrm{Tr}(\rho_i^2)$。对 product 态流形应接近 1;若接近 0.5 通常意味着纠缠/流形错配。
\end{itemize}

\subsection{辅助指标(补充)}
\begin{itemize}
  \item \textbf{类别比例对齐}:\texttt{generated\_pred\_frac\_1} vs \texttt{real\_pred\_frac\_1}(越接近越好)
  \\含义:分类器在生成样本上预测为“1”的比例,反映类别先验/覆盖是否对齐。
  \\注意:该指标依赖 qstates$\to$image 解码与分类器本身;在 purity 已对齐时可作为辅助参考。
\end{itemize}

\section{实验结果}
本次 run 结果汇总(来自 \texttt{data/mnist01\_run\_2025-12-31\_000116/gen/*/eval\_report.json}):

\begin{table}[h]
\centering
\begin{tabular}{lrrrrr}
\toprule
Model & natural distance $\downarrow$ & MMD(Z/ZZ) $\downarrow$ & purity $\to 1$ & real\_pred\_frac\_1 & generated\_pred\_frac\_1 \\
\midrule
QDDPM & 0.450205 & 0.134564 & 1.000000 & 0.5355 & 0.4520 \\
QDT  & 0.602211 & 0.142088 & 1.000000 & 0.5355 & 0.3780 \\
QGAN & 0.704443 & 0.160513 & 1.000000 & 0.5355 & 0.6280 \\
\bottomrule
\end{tabular}
\end{table}

\subsection{作图}
\begin{figure}[h]
\centering
\includegraphics[width=0.95\linewidth]{../data/mnist01_run_2025-12-31_000116/plots/quantum_metrics.png}
\caption{量子态主指标对比(natural distance / MMD(Z/ZZ) / purity)}
\end{figure}

\begin{figure}[h]
\centering
\includegraphics[width=0.75\linewidth]{../data/mnist01_run_2025-12-31_000116/plots/class_fraction.png}
\caption{类别比例对齐(分类器预测为 1 的比例)}
\end{figure}

\section{结论与分析}
\begin{enumerate}
  \item \textbf{QDDPM 在量子态分布质量上最好}:
  natural distance:QDDPM $<$ QDT $<$ QGAN;MMD(Z/ZZ):QDDPM $<$ QDT $<$ QGAN。
  说明 QDDPM 生成分布在量子态空间更接近真实分布。

  \item \textbf{三者 purity$\approx 1$,比较是公平的}:
  purity 均接近 1,表明三种模型都已对齐到 product Ry 流形;因此 QGAN 在距离指标上较差并非由于“跑到纠缠流形”的假象。

  \item \textbf{类别先验(辅助结论)}:
  真实 real\_pred\_frac\_1=0.5355。QDDPM 偏低(0.452),QDT 更偏低(0.378),QGAN 偏高(0.628)。就类别比例对齐而言,QDDPM 的偏差较小,但仍未完全对齐。
\end{enumerate}

\section{后续工作}
\begin{itemize}
  \item 对 QGAN:调整对抗训练的平衡(cycles、epochs\_c/epochs\_g、学习率),缓解不稳定与模式偏移。
  \item 对 QDDPM/QDT:结合 per-qubit 的 $\langle Z\rangle$/$\theta$ 分布与 mean/std 诊断进一步定位误差来源。
  \item 评估方面:多 seed 重复实验,报告均值$\pm$方差提升可信度。
\end{itemize}

\section*{复现方式}
\begin{itemize}
  \item 训练/生成/评估(同一次 run 跑完三种模型并出图):\texttt{nohup bash train.sh > train.log 2>\&1 \&}
  \item 单独重画本次 run 的汇总图:
  \\\texttt{conda run -n qml\_gpu python scripts/mnist01\_plot\_run\_reports.py --exp data/mnist01\_run\_2025-12-31\_000116 --out data/mnist01\_run\_2025-12-31\_000116/plots}
\end{itemize}

\end{document}
